\documentclass[12pt] {article}
\usepackage[top=1.75in, bottom=1.75in, left=1.75in, right=1.75in]{geometry}
\usepackage[pdftex]{hyperref}
\hypersetup{
	pdfauthor={Phil Monroe},
	pdftitle={The Low Down on Graphics Pipelines},
	colorlinks}
	
	
\begin{document}

\title{The Low Down on Graphics Pipelines}
\author{Phil Monroe}
\date{Jan. 17, 2011}
% \maketitle

\centerline{
	\Large \bf The Low Down on Graphics Pipelines} %% Paper title

% \medskip

\centerline{\bf Phil Monroe}
% \medskip

\centerline{Jan. 17, 2011}

\bigskip

% Intro ----------------------------------------------------------------------
Graphic processing units (GPUs) are becoming increasingly important in modern computing. Largely driven by video games, today's computer software  demands higher performance and stunning visuals from GPUs. GPUs start out with several different objects and combine them to form an image in a series of stages called the graphics pipeline. For the sake of this article, think of an image where there is a soccer ball sitting on a table inside of a room. This article describes how this scene can be created on computer by going through all of the stages of the GPU pipeline.

% Application ----------------------------------------------------------------
The graphics pipeline begins with the \emph{application} stage where running software interacts with the GPU. The first step of this stage is to build 3 dimensional models of each object that will be present in the scene. In the case of our soccer ball scene, each object is modeled independently. The 3D models of each of the objects are created using a mesh of triangles. To help visualize this mesh, consider the soccer ball. Real soccer balls are patches of hexagons and pentagons sewn together to form a round ball. So why use triangles instead of hexagons and pentagons to create objects? Well it turns out that complex geometries, such as the hexagons and pentagons, can be broken down into triangles. Triangles also have mathematical properties that make processing the models quicker. Each triangle is defined by the three points that connect the sides, which are know as the vertices. To make these meshes of triangles more realistic looking, textures, which are images that seem to surround the 3D model like wrapping paper, are created and will be applied in later stages.

It is important to note that each 3D model is in it's own coordinate space called model space. This means that each model of an object contains absolutely no information about any of the other objects or how they relate spatially. Our soccer ball is floating in it's own universe where the only thing that exists is the soccer ball. Now the application needs to start transferring all of these models to the GPU to assemble into a scene. Along with the models, the application passes information on how the models should relate spatially, the textures they have, where the lights are and at what direction the user is looking at the models.

% Geometry -------------------------------------------------------------------
Now that the GPU knows about all of the objects that will be present, it can start piecing together the scene in what is called the \emph{geometry} stage. The first step in the geometry stage is to combine all of the models into a common world. This is done by using mathematical transformations to convert the objects from model space into what is called world space. After the transformations, all objects lie in the same universe and are spaced properly. 

%% LIGHTING!!!!

It is now time to consider the viewpoint of the user, which is called the camera. Much like a camera in real life, we position the camera in the scene to determine what will be visible to the user. At this point, the GPU can start trimming away triangles outside of the camera's rectangular viewing area in a process called clipping. Clipping also removes rear facing triangles, such as the triangles on the non-visible side of the soccer ball. These triangles are removed to reduce the amount of processing required in future steps.

Now we need to start making the scene two dimensional. This is accomplished by taking each object and projecting the 3D mesh of visible triangles onto a 2D plane, which produces 2D objects at various distances, or depth, away from the camera. Visualize this as if we were creating a cartoon out of construction paper. We would first create each of the characters and objects in a certain frame and then form layers to determine depth within the scene. 


% Rasterization --------------------------------------------------------------
Next, let's consider the output medium that our scene will be displayed on: the screen. Electronic screens are two dimensional grids of colored dots called pixels. Think of them as sheets of graph paper where you can only color in the individual squares to draw a picture. If the squares are large, the image looks very blocky and non distinct. However, if you shrink the size of the squares you get more detail in the image. Screens can pack millions of tiny pixels close together to draw images with very high detail. So how are we going convert layers of 2D triangles into a bunch of pixels? This conversion process is called \emph{rasterization} and is the next stage of our graphics pipeline.

The rasterization process is similar to taking all of the 2D objects in our scene, laying them out on a sheet of graph paper representing the screen, and if the area of the object covers the center of a square on the graphing paper, then the square is considered part of the rasterized object. Each little square within the blocky rasterized object is called a fragment and contains all of the information, such as initial color and opacity, needed to draw that square to the corresponding pixel on the screen. At the end of the rasterization process, every object in the scene will be converted into boxy rasterized objects at varying depths.

% Texture --------------------------------------------------------------------
Immediately following rasterization, the \emph{texture} stage is applied to the rasterized objects to make them appear more realistic. As hinted to earlier, textures are images that are wrapped around objects to provide a skin. While that is a convenient way of thinking of textures, that is not completely accurate. Textures are 2D images that show all sides of an object. Consider a real soccer ball with scuffs and grass stains that we want to appear on our digital soccer ball. A texture would be created that appears as if the ball was cut up and laid flat in one piece. The pixels in the texture can now be overlaid onto the rasterized object to give more realistic coloring. During this process, some mathematical trickery is used to warp the texture to add perspective to the object and make it appear visually 3D.

% Composite ------------------------------------------------------------------
Now that all of our objects have been rasterized and textured to look realistic, we need to combine the layers in the \emph{composition} stage. During composition, the fragments that make up the rasterized objects are combined to form one solid image. The GPU uses a structure called the frame buffer to hold all of the final pixels to be displayed on the screen. One by one, each object's fragments are copied over to the frame buffer. If two fragments map to the same position in the frame buffer, then the fragment corresponding to the closer object will be the fragment placed in the buffer. For transparent objects, the weighted sum of the fragments are added together to allow distant objects to be visible.


% Display --------------------------------------------------------------------
To display the final image, individual pixels are sent to the screen by reading the frame buffer from beginning to end. Starting with the top left corner, each pixel on the screen is updated from left to right and top to bottom until the bottom right corner is reached. The GPU then repeats the whole process over to display the next image. For movies and movements within a world, the entire pipeline is run multiple times per second to create the illusion of continuous movement.


% References  ----------------------------------------------------------------
\clearpage
\bf References
\medskip


\url{	http://en.wikipedia.org/wiki/Graphics_pipeline}

\url{http://www.ericsink.com/wpf3d/5_Triangles.html}

\url{http://www.arcsynthesis.org/gltut/Basics/Intro%20Graphics%20and%20Rendering.html}

\url{http://www.cs.virginia.edu/~gfx/papers/pdfs/59_HowThingsWork.pdf}

\url{https://smartsite.ucdavis.edu/access/content/group/f376a431-7205-4af5-8cb0-b706f206a07b/lectures/intro.pdf}

\url{https://smartsite.ucdavis.edu/access/content/group/f376a431-7205-4af5-8cb0-b706f206a07b/lectures/OpenGL.pdf}


\end{document}
